\documentclass[pdflatex,compress,mathserif]{beamer}

%\usetheme[dark,framenumber,totalframenumber]{ElektroITK}
\usetheme[darktitle,framenumber,totalframenumber]{ElektroITK}

\usepackage[utf8]{inputenc}
\usepackage[T1]{fontenc}
\usepackage{lmodern}
\usepackage[bahasai]{babel}
\usepackage{amsmath}
\usepackage{amsfonts}
\usepackage{amssymb}
\usepackage{graphicx}
\usepackage{multicol}

\newcommand*{\Scale}[2][4]{\scalebox{#1}{$#2$}}%

\title{MATRIKULASI}
\subtitle{MATEMATIKA DASAR}

\author{Mifta Nur Farid}

\begin{document}

\maketitle

\section{Pertidaksamaan Linier}

	\subsection{Interval}

		\begin{frame}
			\frametitle{Interval}
			
		\end{frame}

	\subsection{Penyelesaian Pertidaksamaan}

\section{Fungsi Dan Limit}

	\subsection{Fungsi}

	\subsection{Limit}

\section{Trigonometri}

\section{Turunan}

\section{Integral}

	\subsection{Integral Tak Tentu}

	\subsection{Integral dengan Substitusi}
	
	\subsection{Integral Tentu}

\end{document}
